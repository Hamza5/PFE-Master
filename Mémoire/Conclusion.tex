\chapter*{Conclusion générale}

Dans ce mémoire nous avons présenté notre projet qui est l'estimation de la distance
à partir des images en utilisant l'apprentissage automatique auto-supervisé.

Afin d'atteindre notre but, nous avons commencé par la construction d'une machine
électronique complexe constituée de plusieurs modules et de différents composants.
Elle est capable de se déplacer sur un terrain régulier, de mesurer
les distances des objets par rapport à son corps, et de prendre des images dans son
champ de vision.

Cette machine est commandée pour naviguer dans un espace fermé contenant éventuellement
des objets ayant des surfaces solides. En se déplaçant dans l'environnement, elle
prend périodiquement une photo et les distances des obstacles visibles. Les images
et les distances respectives sont sauvegardées dans un ensemble de données.

Par la suite, l'ensemble de données est fragmenté en deux : le grand sous-ensemble
est utilisé pour effectuer l'apprentissage du modèle, et l'autre est réservé
comme données de validation. En même temps, des réseaux de neurones convolutionnels
sont construits d'une telle structure permettant de réaliser l'objectif de ce projet.
Les modèles sont alimentés par les données d'apprentissage et sont testés par leurs
performances sont calculés en utilisant les données de validation jusqu'à atteindre
leurs limites d'amélioration.

Comme les résultats obtenus n'étaient pas très bonnes, nous avons pensé à des
idées qui peuvent perfectionner les modèles. Cependant, en raison de manque du temps et
de ressources humaines et matérielles, nous n'avons pas pu les réaliser et nous
les laissons donc comme perspectives.

Une idée est l'ajout plus de capteurs ultrasoniques et de meilleur qualité, ainsi
que la désactivation de capture au moment de déplacement du robot, et la réduction de
sa fréquence dans les autres moments. Cela permet de réduire le nombre de
distances fausses et les images floues causées par la vibration de la machine lors de la prise.

Une autre consiste à collecter beaucoup plus de données (des dizaines de milliers)
dans un seul environnement, puis les équilibrer avant le lancement de l'apprentissage,
car cette opération assure qu'il s'exécute correctement sans que les résultats soient
faussés par les poids des classes.

Une fois que les données deviennent nombreuses, il serait possible d'ajouter plus
que les quatre classes introduite dans ce travail afin d'obtenir un niveau supérieur
de précision. Il serait même possible d'aborder ce problème dans sa nature régressive.

Enfin, nous savons bien que ce travail est loin d'être parfait, mais il est le
premier -à notre connaissance- qui combine deux aspects : d'une part, la construction de la
machine permettant de collecter les données à l'aide \underline{des capteurs ultrasoniques}
et d'une camera ; et d'autre part, l'utilisation de ces données pour faire l'apprentissage
de nos propre modèles. Nous souhaitons que ce travail soit une initiative pour d'autres
recherches dans ce sujet.
