\chapter{Concepts de base}

\section*{Introduction}

Ce chapitre représente une brièvement la robotique, et
l'apprentissage automatique. Il contient quelques définitions et explications
concernant les notions principales de robotique et d'apprentissage automatique
qui sont requises pour ce sujet.

\section{Robotique}

La \textbf{robotique}\footnote{\url{https://en.wikipedia.com/wiki/Robotics}}
est une discipline qui fait partie de la mécanique, de
l'électrique, et l'informatique en même temps. Elle s'intéresse à la conception,
à la construction et au fonctionnement des \emph{robots}, ainsi que la création
des systèmes informatiques qui les contrôlent.

\subsection{Le robot}

Le mot \textbf{robot} refère à un agent électromécanique autonome ou
semi-autonome, guidé par un programme informatique ou un circuit électronique.

Ce terme a été publiquement utilisé pour la première fois par l'écrivain tchèque
Karel Čapek en 1920 dans sa pièce de théâtre \emph{R.U.R.
(Rossum's Universal Robots)}.

Il existe plusieurs types de robots répondant à différents fonctions. Nous
citons les plus importants.

\begin{itemize}
  \item Les robots mobiles ont la possibilité de se déplacer dans leur
  environnement pour accomplir leurs tâches.
  \item Les robots industriels sont construits à partir de bras jointés et
  sont utilisés dans les lignes de production.
  \item Les robots de service ont un rôle proche de celui des robots
  industriels, sauf qu'ils sont utilisés pour fournir des services autres que
  la production.
  \item Les robots éducationnels sont utilisés comme des assistants
  éducationnels pour les enseignants afin d'aider les élèves à mieux comprendre
  des concepts mathématiques, physiques, électroniques et programmatifs.
\end{itemize}

La structure d'un robot comporte trois éléments de point de vue construction.

\begin{itemize}
  \item \emph{Sur le plan mécanique}, tous les robots sont conçus pour leurs
  permettre d'accomplir leurs
  tâches. Cette construction dépend du type de la machine et de l'environnement
  où elle fonctionne. Elle définit la forme extérieure et la structure interne
  du robot.
  \item \emph{Sur le plan électrique}, un robot a besoin de composants
  électriques qui lui offrent la possibilité de
  contrôler son corps mécanique et de capturer les caractéristiques de son
  environnement.
  \item \emph{Sur le plan programmatif}, chaque robot doit avoir un certain
  niveau de programmation qui le permet de
  savoir comment réaliser une opération ou prendre une décision. Cette
  programmation peut être réalisée par une combinaison des circuits intégrés dans
  les cas simples, ou à l'aide d'une série d'instructions exécutées sur un
  microprocesseur dans les cas les plus complexes.
\end{itemize}

\subsection{Les modules électroniques}
Un module est un circuit spécialisé pour réaliser une tâche donnée. Chaque
module contient tous les composants électroniques nécessaires pour son
fonctionnement et sa connexion avec les autres modules.

\begin{description}
  \item[\emph{Le microcontrôleur}] est un module électronique programmable ayant une capacité
  limitée de traitement de données. Il inclut un microprocesseur, une mémoire
  centrale et une mémoire secondaire à lecture seule mais reinitialisable.
  Il peut lire les données reçues des autres modules et leurs envoyer des commandes
  pour contrôler leurs fonctionnement à travers ses ports. Chaque système embarqué
  contient au moins un microcontrôleur.
  \item[\emph{Les capteurs}] sont des modules qui permettent d'obtenir des données sur
  l'environnement sous forme d'un signal électrique numérique ou analogique.
  Parmi ces capteurs, on trouve les capteurs de la température qui mesure le degré
  de la température de l'environnement où il se trouve. Il y a aussi les capteurs
  ultrasoniques permettant d'envoyer et de recevoir un ultrason. Il sont utilisés
  fréquemment pour le calcul de la distance en mesurant le temps écoulé entre
  l'envoi et la réception du signal.
  \item[\emph{Les moteurs}] sont les composants qui transforment l'énergie électrique en
  mouvement mécanique. Ils tournent dans un seul sens et à une vitesse qui
  sont déterminés respectivement par le sens et l'intensité du courant électrique.
\end{description}

\section{Apprentissage automatique}

L'apprentissage automatique (\emph{machine learning}) est une filière de
l'intelligence artificielle qui s'intérresse aux techniques de conception des
modèles ayant la capacité d'apprendre à réaliser une tâche sans les programmer
explicitement. Il y existe de nombreuses méthodes, chacune appartenant à
une des trois classes :
\emph{l'apprentissage supervisé}, \emph{l'apprentissage non supervisé}, et
\emph{l'apprentissage par renforcement}.
Pour le présent travail, c'est l'apprentissage supervisé qui est nécessaire.

\subsection{Apprentissage supervisé}

C'est un mode permettant de trouver une approximation d'une fonction en utilisant
les données avec les résultats attendus de cette fonction. Un modèle approximant
la fonction sera généré en utilisant une des techniques de ce mode. Une fois que
le modèle est entraîné, il serait capable d'opérer sur de nouvelles données qui
n'ont pas été vues lors de l'apprentissage. Parmi ces techniques on trouve
\emph{les réseaux de neuronnes artificiels} auxquelles nous avons recours dans
ce projet.

C'est une technique inspirée du cerveau humain. Son architecture est composée de
plusieurs \emph{couches} dont chacune contient plusieurs unités de calcul dites
\emph{perceptrons}.
Il existe des connexions entre les perceptrons. Le nombre et le type de ces
connexions dépend du type du réseau et il y a plusieurs types de réseaux de
neuronnes possibles, chaque type étant adapté a un type de problèmes.

Dans les
problèmes liés aux images, le type préféré est celui des \emph{réseaux convolutionels}
dont nous nous servirons et que nous détaillons au point suivant.

\section{Les réseaux convolutionels}

Ce type de réseaux se compose de trois couches de
types différents : les \emph{couches convolutionelles}, les
\emph{couches de regroupement (mise en commun)} et les
\emph{couches entièrement connectées (denses)}.

\subsection{Couches convolutionelles}

Chaque perceptron dans une couche de convolution est relié seulement à un petit
sous ensemble de perceptrons de la couche précédente. Cet ensemble représente
une région rectangulaire dans une image. L'image peut être l'originale ou
bien la résultante des opérations des couches précédentes. La fonction générant
chaque région est appelée le \emph{filtre}. Ce type de couches est généralement
placé au début et au milieu du réseau.

Chaque couche convolutionelle nécessite des hyperparamètres qui doivent
être fixés dans la conception : la taille spatiale du filtre $F$ et la
profondeur du filtre $K$, le pas de la convolution $S$, le rembourrage par zero
$P$, et fonction d'activation.

\begin{description}
  \item[\emph{La taille spatiale du filtre $F$}] sert à
  spécifier le nombre d'entrées pour chaque perceptron,
  \emph{\textbf{la profondeur du filtre $K$}} spécifie
  le nombre de matrices de poids partagés.
  \item[\emph{Le pas de la convolution $S$}] est la différence entre le bord d'un
  filtre et le même bord du filtre du perceptron successif.
  \item[\emph{Le rembourrage par zero $P$}] permet d'entourer
  l'entrée par des pixels nuls avant d'appliquer la convolution pour maintenir
  sa taille.
  \item[\emph{La fonction d'activation}] qui, à la sortie de chaque couche, permet grâce
  à une fonction mathématique de modifier les valeurs avant de les transmettre à la
  couche suivante.
\end{description}

\subsection{Couches de regroupement}

Le rôle de ces couches est de réduire la taille spatiale du résultat des couches
précédentes en gardant l'information essentielle. Cette opération est souvant
appliquée en utilisant la valeur maximale entre les pixels ou par le calcul de
la moyenne. Ces couches sont généralement placés après les couches
convolutionelles.

Les hyperparamètres requis pour cette couche sont les mêmes que ceux des couches
convolutionelles auxquelles il faut ajouter la La fonction du regroupement
(maximum ou moyenne).

\subsection{Couches entièrement connectées}

C'est le type classique des couches dans un réseau de neuronnes artificiels.
Chaque perceptron dans ce type est connecté à tous les perceptrons de la couche
précédente. Ces couches sont utilisées à la fin du réseau.

Les hyperparamètres nécessaires pour ce type :

\begin{itemize}
  \item le nombre de perceptrons $K$,
  \item la fonction d'activation.
\end{itemize}

\section*{Conclusion}

Ce chapitre a été consacré aux notions de la robotique et l'intelligence
artificielle requises pour la compréhension du sujet. Le chapitre suivant fera
la description matérielle et logicielle de la réalisation présentée.
