\begin{titlepage}

  \vspace*{\fill}

  \section*{\LARGE Résumé}

  Ce travail aborde le sujet d'estimation des distances des objets dans des images
  RGB en utilisant l'approche de l'apprentissage automatique auto-supervisé.
  Il est constitué de deux parties principales.

  Dans la première partie, il s'agit de fabriquer une machine mobile équipée de
  capteurs ultrasoniques et d'un appareil disposant d'une caméra. Son rôle
  est de se déplacer dans un environnement fermé en prenant des images et les
  distances correspondantes des obstacles.

  La deuxième partie consiste à concevoir des réseaux de neurones convolutionnels
  puis effectuer leur apprentissage en utilisant les données issues de la première
  partie après avoir appliqué les prétraitements nécessaires. Elle inclut aussi
  le test de performances de ces modèles ainsi que leur implémentation dans un
  logiciel permettant de les exploiter en utilisant d'autres données.

  \textbf{Mots clefs :} estimation de distance, vision par ordinateur,
  électronique, robot, apprentissage automatique, réseau de neurones, intelligence artificielle.

  \vspace{0.5em}

  \section*{\LARGE Abstarct}

  This work address the subject of estimating the distances of objects in RGB pictures
  using the self-supervised machine learning approach. It is made of two main parts.

  In the first part, it is a matter of manufacturing a mobile machine equipped with
  ultrasonic sensors and a device having a camera. Its role is to move in a closed
  environment while taking pictures and the corresponding distances of the obstacles.

  The second part consists of designing convolutional neural networks then performing
  their training using the data issued from the first part after applying the
  necessary preprocessing. It includes also the performances test of these models
  as well as their implementation in a software allowing to exploit them using
  other data.

  \textbf{Keywords :} distance estimation, computer vision,
  electronics, robot, machine learning, neural networks, artificial intelligence.

  \vspace*{\fill}

\end{titlepage}
