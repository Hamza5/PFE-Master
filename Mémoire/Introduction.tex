\chapter*{Introduction générale}

L'être humain, en utilisant ses yeux et son cerveau, est capable de distinguer facilement
entre les niveaux de distances de différents objets qui se trouvent dans une scène.
Par contre, les machines programmables, comme les ordinateurs, ne disposent d'aucune
méthode algorithmique qui leur permettent de ce faire à partir des images bidimensionnelles
reçues à travers les caméras ordinaires.

Ce projet se situe dans le contexte de l'extraction des profondeurs d'une image 2D,
ou de l'estimation de la distance d'un objet par rapport à la camera, qui font partie des problèmes appartenant à une catégorie
connue dans le domaine de la vision par ordinateur. Plusieurs approches ont été
proposées pour les résoudre, chacune ayant ses propres contraintes, avantages, et
inconvénients.

La méthode utilisée dans ce travail est basée sur \keyword{l'apprentissage automatique
supervisé} dont le principe est d'utiliser un ensemble de données contenant les
images et les profondeurs correspondantes. Son avantage majeur est de pouvoir
produire un modèle permettant d'estimer les profondeurs sans avoir à implémenter
les détails du traitement explicitement.

Cependant, ces méthodes nécessitent
un nombre considérable de données afin de pouvoir fonctionner correctement.
L'obtention des données et la correspondance entre les entrées et les sorties
est un travail qui se fait généralement de façon manuelle dans la majorité des tâches
et qui demande un effort important du côté des constructeurs de l'ensemble. C'est la
raison pour laquelle la majorité des travaux qui concernent l'apprentissage automatique
utilisent des ensembles prêts, préparés dans des travaux antérieurs (par exemple
MNIST~\cite{lecun2010mnist} et CIFAR-10~\cite{krizhevsky2009learning}).

En effet, dans certaines tâches, la génération de cet ensemble peut être automatisée
et donc toutes les opérations précédentes sont effectuées par des programmes au lieu
des personnes. Cela fait partie des méthodes de \keyword{l'apprentissage
automatique auto-supervisé} qui n'est qu'un cas spécial de celui supervisé où
l'ensemble de données est construit automatiquement.

C'est dans ce contexte que se situe notre travail, qui combine deux
grandes tâches : d'abord, générer un ensemble de données contenant les images et les distances
des obstacles correspondantes après avoir construit une machine mobile permettant
de ce faire; ensuite, concevoir un modèle d'apprentissage automatique et utiliser
l'ensemble de données pour effectuer son entraînement.

Contrairement aux travaux
antérieurs, la première tâche est réalisée ici en utilisant des capteurs ultrasoniques
et non pas des capteurs lasers. Les modèles conçus dans la deuxième tâche sont
des réseaux de neurones convolutionnels.

Nous avons organisé le mémoire comme suit.
D'abord, un chapitre contenant des concepts de base et des notions générales dans
les domaines de la robotique, la vision par ordinateur, et l'apprentissage automatique.
Ensuite, un deuxième chapitre consacré à l'exposition de quelques travaux précédents
autour de notre sujet. Le troisième présente les parties
matérielles et logicielles de notre machine et explique son fonctionnement.
Le quatrième décrit les structures de nos données et les
architectures de nos réseaux convolutionnels. Le dernier présente
les exécutions d'apprentissage que nous avons lancées et les résultats recueillis
de nos tests.
