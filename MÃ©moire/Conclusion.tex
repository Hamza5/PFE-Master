\chapter*{Conclusion générale}

Dans ce mémoire nous avons présenté notre projet qui est l'estimation de la distance
à partir des images en utilisant l'apprentissage automatique auto-supervisé.

Pour cela, nous avons d'abord construit d'une machine
électronique complexe constituée de plusieurs modules et de différents composants.
Elle est capable de se déplacer sur un terrain régulier, de mesurer
les distances des objets par rapport à son corps, et de prendre des images dans son
champ de vision.

Cette machine est commandée pour naviguer dans un espace fermé contenant éventuellement
des objets ayant des surfaces solides. En se déplaçant dans l'environnement, elle
prend périodiquement une photo et les distances des obstacles visibles. Les images
et les distances respectives sont sauvegardées dans un ensemble de données.

Cet ensemble de données est ensuite fragmenté en deux : le grand sous-ensemble
est utilisé pour effectuer l'apprentissage du modèle, et l'autre est réservé
comme données de validation. En même temps, des réseaux de neurones convolutionnels
sont construits à partir d'une structure telle qu'elle permette de réaliser l'objectif du projet.
Les modèles sont alimentés par les données d'apprentissage et leurs
performances sont calculées en utilisant les données de validation jusqu'à atteindre
leurs limites d'amélioration.

Comme les résultats obtenus n'étaient pas très bons, nous avons réfléchi aux moyens
permettant de perfectionner les modèles. Mais, en raison du manque de temps et
de ressources humaines et matérielles, nous n'avons pas pu les réaliser et nous
les proposons ici comme des perspectives d'étude ultérieure.

L'une de nos idées est l'ajout de nouveaux capteurs ultrasoniques et de meilleure qualité, ainsi
que la désactivation de capture au moment du déplacement du robot et la réduction de
sa fréquence dans les autres moments. Cela permettrait de réduire le nombre de
distances fausses et les images floues causées par la vibration de la machine lors de la prise.

Une autre piste serait de collecter beaucoup plus de données (des dizaines de milliers)
dans un seul environnement, puis de les équilibrer avant le lancement de l'apprentissage,
car cette amélioration assurerait que l'apprentissage s'exécute correctement, sans que les résultats soient
faussés par les poids des classes.

Les données étant ainsi beaucoup plus nombreuses, il serait possible d'introduire plus
que les quatre classes actuelles afin d'obtenir un niveau supérieur
de précision. Il serait même possible d'aborder ce problème dans sa nature régressive.

Ce travail qui demande complément et approfondissement, est le
premier à notre connaissance qui combine la construction d'une machine
permettant de collecter les données à l'aide \underline{des capteurs ultrasoniques}
et d'une camera et l'apprentissage des modèles (qui sont les notres)
par l'utilisation de ces données. Nous souhaitons que ce travail soit une initiative pour d'autres
recherches dans ce sujet.
