\chapter{Concepts de base}

\section{Introduction}

Ce chapitre représente une courte introduction concernant la robotique, et
l'apprentissage automatique. Ils contient quelques définitions et explications
sur les notions principales dans ces domaines qui sont requises pour ce sujet.

\section{Robotique}

\subsection{Définition}

La \textbf{robotique}\footnote{\url{https://en.wikipedia.com/wiki/Robotics}}
est une discipline qui fait partie de la mechanique,
l'électrique, et l'informatique en même temps. Elle s'intéresse à la conception,
la construction et le fonctionnement des \emph{robots}, ainsi que la création des
systèmes informatiques qui les controlent.

\subsection{Robot}

Le mot \textbf{robot} refère à un agent électromechanique autonome ou
semi-autonome, guidé par un programme informatique ou un circuit électronique.

Ce terme a été publiquement utilisé pour la première fois par l'écrivain tchèque
Karel Čapek en 1920 dans sa pièce de théâtre \emph{R.U.R.
(Rossum's Universal Robots)}.

Il existe plusieurs types de robots, parmi eux il y a les :

\begin{itemize}
  \item Robots mobiles : qui ont la possibilité de se déplacer dans leur
  environnement pour accomplir leurs tâches.
  \item Robots industriels : qui sont construits à partir des bras jointés. Ils
  sont utilisés dans les lignes de production.
  \item Robots de service : leur définition ressemble à celle des robots
  industriels sauf qu'ils sont utilisés pour fournir des services autres que
  la production.
  \item Robots éducationnels : qui sont utilisés comme des assistants
  éducationnels pour les enseignants afin d'aider les enfants à mieux comprendre
  des concepts mathématiques, physiques, électronique et de programmation.
\end{itemize}

\subsection{Les aspects robotiques}

Malgré les différences existantes entre les types de robots, ils partagent tous
les même principes en point de vue de leur construction.

\subsubsection{L'aspect mécanique}

Tous les robots ont une construction mécanique qui lui permet d'accomplir sa
tâche. Cette construction dépend du type de la machine et l'environnement où
elle fonctionne. Elle définit la forme extérieure et la structure interne du
robot.

\subsubsection{L'aspect électrique}

Un robot a besoin des composants électriques qui lui offre la possibilité de
contrôler son corps mécanique et de capturer les caractéristiques de son
environnement.

\subsubsection{L'aspect programmation}

Chaque robot doit avoir un certain niveau de programmation qui le permet de
savoir comment réaliser une opération ou prendre une décision. Cette
programmation peut être réalisée par une combinaison des circuits intégrés dans
les cas simples, ou à l'aide d'une série des instructions exécutées sur un
microprocesseur dans les cas les plus complexes.

\subsection{Modules électroniques}
Un module est un circuit spécialisé pour réaliser une tâche donnée. Chaque
module contient tous les composantes électroniques nécessaire pour son
fonctionnement et sa connexion avec les autres modules.

\subsubsection{Microcontrôleur}
Le microcontrôleur est un module électronique programmable ayant une capacité
limitée de traitement des données. Il inclut un microprocesseur, une mémoire
centrale et une mémoire secondaire à lecture seule mais reinitialisable.
Il peut lire les données reçus des autres modules et leurs envoyer des commandes
pour contrôler leurs fonctionnement à travers ses ports. Chaque système embarqué
contient au moins un microcontrôleur.

\subsubsection{Capteurs}
Les capteurs sont des modules qui permetent d'obtenir des données sur
l'environnement sous forme d'un signal électrique (numérique ou analogique).
Parmi ces capteurs, on trouve les capteurs de la temperature qui mésure le degré
de la temperature de l'environnement où il se trouve. Il y a aussi les capteurs
ultrasonique permettant d'envoyer et recevoir un ultrason, il sont utilisés
fréquemment pour le calcul de la distance en mésurant le temps écoulé entre
l'envoi et la réception du signal.

\subsubsection{Moteurs électriques}
Les moteurs sont les composants qui transforment l'énergie électrique en
mouvement mécanique. Ils tournent dans un seul sens à une vitesse dont ces deux
sont déterminés respectivement par le sens et l'intensité du courant électrique.

\section{Apprentissage automatique}

L'apprentissage automatique (\emph{Machine learning}) est une fillière de
l'intelligence artificielle qui s'interrese aux techniques de conception des
modèles ayant la capacité d'apprendre à réaliser une tâche sans les programmer
explicitement. Il y existe de nombreux méthodes où chacune appartient à
une des trois classes principales :
\emph{l'apprentissage supervisé}, \emph{l'apprentissage non supervisé}, et
\emph{l'apprentissage par renforcement}.

\subsection{Apprentissage supervisé}

C'est un mode permettant de trouver une approximation d'une fonction en utilisant
les données avec les résultats attendus de cette fonction. Un modèle appromximant
la fonction sera généré en utilisant une des techniques de ce mode. Une fois le
modèle est entraîné, il serait capable d'opérer sur de nouvelles données qui
n'ont pas été vu lors de l'apprentissage. Parmi ces techniques on trouve
\emph{les réseaux de neuronnes artificiels}.

\subsection{Réseaux de neuronnes artificiels}

C'est une technique inspirée du cerveau humain. Son architecture est composée de
plusieurs \emph{couches} dont chacune contient plusieurs unités de calcul dites
les \emph{perceptrons}.

Il existe des connexions entre les perceptrons. Le nombre et le type des
connexions dépend du type du réseau. Il y a plusieurs types de réseaux de
neuronnes, dont chaque type est mieux adapté a un type de problèmes. Dans les
problèmes liés aux images, le type préféré est \emph{les réseaux convolutionels}

\subsection{Réseaux convolutionels}

Ce type de réseaux se compose de plusieurs couches de qui peuvenet être de trois
types différents : \emph{couches convolutionelles},
\emph{couches de regroupement (mise en commun)} et les
\emph{couches entièrement connectées (denses)}.

\subsubsection{Couches convolutionelles}

Chaque perceptron dans une couche de convolution est relié seulement à un peitit
sous ensemble des perceptrons de la couche précédente. Cet ensemble représente
une région rectangulaire dans une image. L'image peut être celle l'originale ou
bien le résultat des opérations des couches précédentes. La fonction générant
chaque région est appelée le \emph{filtre}. Ce type de couches est généralement
placé au début et au milieu du réseau.

Chaque couche convolutionelle nécessite un nombre de hyperparamètres qui doivent
être fixés dans la conception :

\begin{description}
  \item[La taille spatiale du filtre $F$ et sa profondeur $K$.] La taille sert à
  spécifier le nombre d'entrées pour chaque perceptron. La profondeur spécifie
  le nombre de matrices de poids partagés.
  \item[Le pas de la convolution $S$.] C'est la différence entre le bord d'un
  filtre et le même bord du filtre du perceptron successif.
  \item[L'utilisation de rembourrage par zero $P$.] Il est possible d'entourer
  l'entrée par des pixels nuls avant d'appliquer la convolution pour maintenir
  sa taille.
  \item[La fonction d'activation.] La sortie de chaque couche est passée par une
  fonction mathématique qui modifie les valeurs avant de les transmettre à la
  couche suivante.
\end{description}

\subsubsection{Couches de regroupement}

Le rôle de ces couches est de réduire la taille spatiale du résultat des couches
précédentes en gardant l'information essentielle. Cette opération est souvant
appliquée en utilisant la valeur maximale entre les pixels ou par le calcul de
la moyenne. Ces couches sont généralement placés après les couches
convolutionelles.

Les hyperparamètres requis pour cette couche sont :

\begin{itemize}
  \item La fonction du regroupement (maximum ou moyenne).
  \item La taille spatiale du filtre $F$.
  \item Le pas du mouvement $S$.
  \item L'utilisation de rembourrage par zero $P$.
  \item La fonction d'activation.
\end{itemize}

\subsubsection{Couches entièrement connectées}

C'est le type classique des couches dans un réseau de neuronnes artificiels.
Chaque perceptron dans ce type est connecté à tous les perceptrons de la couche
précédente. Ces couches sont utilisées à la fin du réseau.

Les hyperparamètres nécessaires pour ce type :

\begin{itemize}
  \item Le nombre de perceptrons $K$.
  \item La fonction d'activation.
\end{itemize}

\section{Conclusion}

Dans ce chapitre, j'ai exposé les notions de la robotique et l'intelligence
artificielle requises pour la compréhension du sujet. Dans le chapitre suivant
je présenterai la description matérielle et logicielle de ma réalisation.
